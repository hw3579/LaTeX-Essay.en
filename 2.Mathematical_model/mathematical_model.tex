\section{Background and literature review}
\label{sec:problem_description}

\subsection*{Deep Learning in Biological Tissue Analysis}

\textbf{Recent Developments:} Deep learning, a subset of machine learning, has significantly transformed the way biological images are analyzed and interpreted. This transformation is underpinned by a collection of algorithms capable of deciphering the content of images, which is especially pertinent in cellular image analysis.\cite{ref1}

\textbf{Key Applications:} The field has seen progress in various applications like image classification, segmentation, object tracking, and augmented microscopy. These advancements have rendered complex analyses more routine and enabled the execution of experiments that were previously impossible.\cite{ref2}

\subsection*{Deep Learning in Material Science for Tool Wear Assessment}

\textbf{Industry Revolution:} Deep learning has revolutionized the field of materials degradation by providing robust methods to interpret large quantities of data, crucial for detecting and modeling material deterioration.\cite{ref3}

\textbf{Detection Techniques:} The detection of degradation in materials like steel, concrete, and composites is essential to prevent failure and ensure safety. This includes both direct detection methods (like visual inspection for corrosion or cracks) and indirect methods (like ultrasonic testing)\cite{ref4}

\FloatBarrier
