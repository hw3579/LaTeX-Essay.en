\section{Example}
\label{sec:Example}
\textbf{This section includes some examples that are not commonly used}
\subsection{Enumerate}
\begin{enumerate}
	\item 1
	\item 2
	\item 3
	\item 4
\end{enumerate}
itemize
\begin{itemize}
	\item 1
	\item 2
	\item 3
	\item 4
\end{itemize}
\newpage


\subsection{Table}

\subsubsection{Tables side by side}
\begin{minipage}[c]{0.5\textwidth}
	\captionof{table}{Difference of Mild Steel}
	\label{T 4.1}
	\centering
	\begin{tabular}{@{}ccc@{}}
		\toprule
		Loading & Difference & Difference rate \\ \midrule
		50N     & 0.01906 mm & 16.5681$\%$     \\
		100N    & 0.03803 mm & 16.5298$\%$     \\
		150N    & 0.05709 mm & 16.5426$\%$     \\ \midrule
		Average &            & 16.55$\%$           \\ \bottomrule
	\end{tabular}
\end{minipage}
\begin{minipage}[c]{0.5\textwidth}
	\captionof{table}{Difference of Alminium}
	\label{T 4.2}
	\centering
	\begin{tabular}{@{}ccc@{}}
		\toprule
		Loading & Difference & Difference rate \\ \midrule
		50N     & 0.03944 mm & 12.1856$\%$     \\
		100N    & 0.07887 mm & 12.1839$\%$     \\
		150N    & 0.11831 mm & 12.1845$\%$     \\ \midrule
		Average &            & 12.18$\%$           \\ \bottomrule
	\end{tabular}
\end{minipage}

\subsubsection{General table}
\begin{table}[htp]
	\centering
	\caption{The value of $C_{L}$}
	\label{T 4.3}
	\setlength{\tabcolsep}{2mm} %单位间距
	\renewcommand\arraystretch{1} %上下间距
	\begin{tabular}{cllllllll}
		\hline
		Value\textbackslash{}Degree &
		\multicolumn{1}{c}{0} &
		\multicolumn{1}{c}{5} &
		\multicolumn{1}{c}{10} &
		\multicolumn{1}{c}{15} &
		\multicolumn{1}{c}{17.5} &
		\multicolumn{1}{c}{20} &
		\multicolumn{1}{c}{22.5} &
		\multicolumn{1}{c}{25} \\ \hline
		$C_L$ &
		0.034 &
		-0.378 &
		-0.658 &
		-0.892 &
		-0.954 &
		-0.747 &
		-0.717 &
		-0.702 \\ \hline
	\end{tabular}
\end{table}
\newpage

\subsection{Picture}

\subsubsection{Pictures side by side}
\textbf{Images side-by-side, each with its own subheading but sharing large headings and tags}
\begin{figure}[htp]
	\centering
	\subfloat[50N loading Mild Steel]{\includegraphics[width=9cm]{"Example/figures/Steel50"}}
	\subfloat[100N loading Mild Steel]{\includegraphics[width=9cm]{"Example/figures/Steel100"}}
	\\
	\centering
	\subfloat[150N loading Mild Steel]{\includegraphics[width=9cm]{"Example/figures/150Steel"}}
	\subfloat[50N loading Aluminium]{\includegraphics[width=9cm]{"Example/figures/Al50"}}
	\\
	\centering
	\subfloat[100N loading Aluminium]{\includegraphics[width=9cm]{"Example/figures/Al100"}}
	\subfloat[150N loading Aluminium]{\includegraphics[width=9cm]{"Example/figures/Al150"}}
	\caption{Results of experiments with Steel and Aluminium} %图片标题
	\label{F 1.1}
\end{figure}

\subsubsection{picture name adjust}
\begin{figure}[H]
	\centering
	\includegraphics[width=1.0\linewidth]{"Example/figures/result"}
	\caption*{}
	\label{T 2.1}
%通过在图片中编辑名字,然后取消命名与自动编号
\end{figure}
\newpage

\subsection{Equation}
\textbf{Editing by Axmath or python pix2tex (cmd input latexocr if you have been install pix2tex in your system )}

\begin{Large}
	\begin{equation}
		\label{E 2.1}
		\left\{ \begin{array}{c}
			\delta _{An\_1}=\frac{P_{50N}L^3}{48E_sI}=\frac{50\times 0.1^3}{48\times 172.6698\times 10^9\times 4.5\times 10^{-11}}=0.1341\times 10^{-3}m\\
			\\
			\delta _{An\_2}=\frac{P_{100N}L^3}{48E_sI}=\frac{100\times 0.1^3}{48\times 172.6698\times 10^9\times 4.5\times 10^{-11}}=0.2681\times 10^{-3}m\\
			\\
			\delta _{An\_3}=\frac{P_{150N}L^3}{48E_sI}=\frac{150\times 0.1^3}{48\times 172.6698\times 10^9\times 4.5\times 10^{-11}}=0.4022\times 10^{-3}m\\
		\end{array} \right. 
		\\
	\end{equation}
\end{Large}



\FloatBarrier % Now figures cannot float above section title



